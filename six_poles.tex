\documentclass[a4paper,12pt]{article}

\usepackage{cmap}
\usepackage[T2A]{fontenc}
\usepackage[utf8x]{inputenc}
\usepackage[english, russian]{babel}

\usepackage{misccorr} % в заголовках появляется точка, но при ссылке на них ее нет
\usepackage{amssymb,amsfonts,amsmath,amsthm}  
\usepackage{indentfirst}
\usepackage[usenames,dvipsnames]{color} 
\usepackage[unicode,hidelinks]{hyperref}
% \hypersetup{%
%     pdfborder = {0 0 0}
% }
\usepackage{makecell,multirow} 
\usepackage{ulem}
\usepackage{graphicx,wrapfig}
\graphicspath{{img/}}
\usepackage{geometry}
\geometry{left=2cm,right=2cm,top=3cm,bottom=3cm,bindingoffset=0cm,headheight=15pt}
\usepackage{fancyhdr} 
\linespread{1.2} 
\frenchspacing 
\renewcommand{\labelenumii}{\theenumii)} 
% \usepackage{caption}
%%%%%%%%%%%%%%%%%%%%%%%%%%%%%%%%%%%%%%%%%%%%%%%%%%%%%%%%%%%%%%%%%%%%%%%%%%%%%%%
%%%%%%%%%%%%%%%%%%%%%%%%%%%%%%%%%%%%%%%%%%%%%%%%%%%%%%%%%%%%%%%%%%%%%%%%%%%%%%%

\def\labauthor{Сарафанов Ф.Г., Платонова М.В.}
\def\labauthors{\labauthor}
\def\labnumber{3}
\def\labtheme{Исследование матриц рассеяния \\[0.4em] волноводных узлов}

%%%%%%%%%%%%%%%%%%%%%%%%%%%%%%%%%%%%%%%%%%%%%%%%%%%%%%%%%%%%%%%%%%%%%%%%%%%%%%%
	%применим колонтитул к стилю страницы
\pagestyle{fancy} 
	%очистим "шапку" страницы
\fancyhead{} 
	%слева сверху на четных и справа на нечетных
\fancyhead[L]{\labauthors} 
	%справа сверху на четных и слева на нечетных
\fancyhead[R]{Отчёт по лабораторной работе №\labnumber} 
	%очистим "подвал" страницы
\fancyfoot{} 
	% номер страницы в нижнем колинтуле в центре
\fancyfoot[C]{\thepage} 

%%%%%%%%%%%%%%%%%%%%%%%%%%%%%%%%%%%%%%%%%%%%%%%%%%%%%%%%%%%%%%%%%%%%%%%%%%%%%%%

\usepackage{float}
\usepackage[mode=buildnew]{standalone}
\usepackage{tikz} 
% \usepackage{subcaption}
\usepackage{tikz,csvsimple}
\usetikzlibrary{scopes}
\usetikzlibrary{%
     decorations.pathreplacing,%
     decorations.pathmorphing,%
    patterns,%
    calc,%
    scopes,%
    arrows,%
    % arrows.spaced,%
}
\makeatletter
\newif\if@gather@prefix 
\preto\place@tag@gather{% 
  \if@gather@prefix\iftagsleft@ 
    \kern-\gdisplaywidth@ 
    \rlap{\gather@prefix}% 
    \kern\gdisplaywidth@ 
  \fi\fi 
} 
\appto\place@tag@gather{% 
  \if@gather@prefix\iftagsleft@\else 
    \kern-\displaywidth 
    \rlap{\gather@prefix}% 
    \kern\displaywidth 
  \fi\fi 
  \global\@gather@prefixfalse 
} 
\preto\place@tag{% 
  \if@gather@prefix\iftagsleft@ 
    \kern-\gdisplaywidth@ 
    \rlap{\gather@prefix}% 
    \kern\displaywidth@ 
  \fi\fi 
} 
\appto\place@tag{% 
  \if@gather@prefix\iftagsleft@\else 
    \kern-\displaywidth 
    \rlap{\gather@prefix}% 
    \kern\displaywidth 
  \fi\fi 
  \global\@gather@prefixfalse 
} 
\newcommand*{\beforetext}[1]{% 
  \ifmeasuring@\else
  \gdef\gather@prefix{#1}% 
  \global\@gather@prefixtrue 
  \fi
} 
\makeatother

\usepackage{booktabs}
\usepackage{pgfplots, pgfplotstable}

\usepackage[outline]{contour}
\usepackage{tocloft}
\renewcommand{\cftsecleader}{\cftdotfill{\cftdotsep}} % for parts
% \renewcommand{\cftchapleader}{\cftdotfill{\cftdotsep}} % for chapters
\usepackage{pgfplots,pgfplotstable,booktabs,colortbl}

% \renewcommand{\arraystretch}{1.5} 

\pgfkeys{/pgf/number format/.cd,
		fixed,  1000 sep={\,}}

\pgfplotstableset{
	% multicolumn names, % allows to have multicolumn names
	% header=has colnames,
	% dec sep align,
	col sep=tab, % the seperator in our .csv file
	% fixed zerofill, 
	% precision=4,
	columns/1/.style={
		column name={1},
		string type,
	},	
	columns/2/.style={
		column name={2},
		string type,
	},	
	columns/3/.style={
		column name={3},
		string type,
	},	
	columns/Skm/.style={
		column name={$S_{km}$},
		string type,
	},	
	columns/S/.style={
		column name={$S_{km}$},
		string type,
		column type = {l},
	},	
	columns/N/.style={
		column name={№},
		precision=0,
		% fixed zerofill, 	
		% % column type/.add={|}{},
	},
	columns/Imax/.style={
		column name={$I_{\max}$},
		dec sep align,
		precision=0,
	},
	columns/Imin/.style={
		column name={$I_{\min}$},
		dec sep align,
		precision=1,
	},
	columns/Gamma/.style={
		column name={$|\Gamma|$},
		dec sep align,
		precision=2,
	},
	columns/phi/.style={
		column name={$\varphi_H$},
		dec sep align,
		precision=2,
	},
	columns/zmin/.style={
		column name={$z_{\min}$, см},
		dec sep align,
		precision=2,
	},
	empty cells with={\textbf{--}},
	every head row/.style={
	before row={\toprule},
	after row={
		\midrule}
		},
	every last row/.style={after row=\bottomrule},
	every row/.style={after row=\midrule}, 
	create on use/Gamma/.style={
	    create col/expr={
	    	(sqrt(\thisrow{Imax}/\thisrow{Imin})-1)/(sqrt(\thisrow{Imax}/\thisrow{Imin})+1)%*\thisrow{ul5}/\thisrow{u}
	    }
	},
	create on use/phi/.style={
	    create col/expr={
	    	4*pi*abs(\thisrow{zmin}-5.145)/5.47-pi
	    }
	},
	columns={N,Skm,1,2,3,Imax,Imin,zmin,Gamma,phi,S},		
	% dec zerofill
	% fixed,fixed zerofill,
	% precision=3
	% every even column/.style={%
	% 	% column type/.add={>{\columncolor[gray]{.8}}}{}
	% },
	% every even row/.style={%
	% 	before row={\rowcolor[gray]{0.95}},
	% },	
	% every head row/.style={
 %        before row={
 %        	& & \multicolumn{3}{c}{Номер входа} & \\ \toprule
 %        },
 %        after row=\midrule
	% },
	}%

\pgfplotsset{compat=newest}
\usepackage{physics}
\usepackage{mathtools}
\mathtoolsset{showonlyrefs=true}
\newcommand\Smat{\hat { \mathbf { S } }}


\begin{document}

\begin{titlepage}
\begin{center}

{\textsc{Нижегородский государственный университет имени Н.\,И. Лобачевского}}
\vskip 2pt \hrule \vskip 3pt
{\textsc{Радиофизический факультет}}

\vfill


{{\LARGE Отчет по лабораторной работе №\labnumber}\vskip 12pt {\Huge \bfseries \labtheme}}

	
\vspace{2cm}
{\large Работу выполнили студенты \\[-0.25em] 430 группы радиофизического факультата \\[0.5em] {\Large \bfseries \labauthor}}

% \vspace{0.5cm}
% {e-mail: sfg180@yandex.ru}

% \vspace{2cm}

\end{center}

\vfill
	
% \begin{flushright}
% 	{Выполнили студенты 430 группы\\ \labauthor}%\vskip 12pt Принял:\\ Менсов С.\,Н.}
% \end{flushright}
	
% \vfill
	
\begin{center}
	{Нижний Новгород, \today}
\end{center}

\end{titlepage}

\tableofcontents
\newpage

\section*{Введение}
\addcontentsline{toc}{section}{Введение}
\label{sec:input}

В данной работе изучаются с помощью матричного анализа волноводные узлы -- шестиполюсники. У них с помощью измерительной линии измеряются величины, позволяющие рассчитать коэффициенты матрицы рассеяния шестиполюсников $S_{km}$. 

На основе рассчитанной матрицы рассеяния $S$ конкретного шестиполюсников можно попытаться решить обратную задачу: сделать на основе полученных данных предположение о возможных конструктивных вариантах волноводных узлов, находящихся внутри шестиполюсников.

\section{Теоретические сведения}
\subsection{Матрица рассеяния шестиполюсника}

Рассмотрим трехплечий волноводный узел (шестиполюсник), изображенный на рис. 1. В каждом плече выберем \textbf{плоскость отсчета} (сечение), в котором будем находить отношения амплитуд полей отраженной и падающей волн.

\begin{figure}[h!]
	\centering
	\includegraphics[scale=1.5]{ris/ris1}
	\caption{Схема трехплечего узла (шестиполюсника)}
	\label{fig:figure1}
\end{figure}

Обозначим комплексные амплитуды полей входящих (падающих) в узел волн через $U_m^+$, а амплитуды выходящих (отраженных) волн через $U_k^-$. 
Величины $U_k^-$ зависят от амплитуд и фаз полей волн, входящих во все плечи узла, причем эти зависимости являются линейными в силу линейности уравнений Максвелла (предполагается, что нелинейных элементов в узле нет). 
Связь между амплитудами полей в плечах узла записывается в виде:
\begin{gather}
	\label{eq:usu}
	{ U _ { 1 } ^ { - } = S _ { 11 } U _ { 1 } ^ { + } + S _ { 12 } U _ { 2 } ^ { + } + S _ { 13 } U _ { 3 } ^ { + } } \\ 
	{ U _ { 2 } ^ { - } = S _ { 21 } U _ { 1 } ^ { + } + S _ { 22 } U _ { 2 } ^ { + } + S _ { 23 } U _ { 3 } ^ { + } } \\
	{ U _ { 3 } ^ { - } = S _ { 31 } U _ { 1 } ^ { + } + S _ { 32 } U _ { 2 } ^ { + } + S _ { 33 } U _ { 3 } ^ { + } } 
\end{gather}
где $S_{km}$ ---  комплексные коэффициенты, характеризующие волноводный узел. 

Систему уравнений \eqref{eq:usu} удобно записать в матричной форме
\begin{equation}
	\left( \begin{array} { c } { U _ { 1 } ^ { - } } \\ { U _ { 2 } ^ { - } } \\ { U _ { 3 } ^ { - } } \end{array} \right) = \Smat \left( \begin{array} { l } { U _ { 1 } ^ { + } } \\ { U _ { 2 } ^ { + } } \\ { U _ { 3 } ^ { + } } \end{array} \right), \quad\text{где}\quad
	\Smat = \left( \begin{array} { c c c } { S _ { 11 } } & { S _ { 12 } } & { S _ { 13 } } \\ { S _ { 21 } } & { S _ { 22 } } & { S _ { 23 } } \\ { S _ { 31 } } & { S _ { 32 } } & { S _ { 33 } } \end{array} \right)
\end{equation}

Матрица $\Smat$ называется матрицей рассеяния, или $S$-матрицей (от англ. scattering -- рассеяние).

Из определения элементов матрицы рассеяния следует, что для пассивных узлов, не обладающих свойством усиления мощности, модули коэффициентов передачи и отражения не могут превышать единицы.
\subsection{Свойства матрицы рассеяния}


\paragraph{Взаимный узел.} Волноводные узлы, в которых отсутствуют элементы с гиротропными свойствами (например, намагниченный феррит), являются взаимными устройствами. Их матрицы рассеяния симметричны относительно главной диагонали

\begin{equation}
	S_{mk}=S_{km}
\end{equation}

Верно и обратное утверждение: если волноводное устройство описывается симметричной матрицей рассеяния, то оно является взаимным.

Отметим, что для взаимных узлов свойства матриц, доказанные для строк, выполняются и для столбцов, и наоборот.

\paragraph{Волноводное устройство без потерь.} Матрица рассеяния волноводного устройства без потерь является унитарной, т.е.
\begin{equation}
	\Smat^T\Smat^*= \hat { \mathbf { I } }
\end{equation}
Можно показать \cite{met}, что для унитарной матрицы выполняются следующие свойства:
\begin{equation}
	\sum _ { m = 1 } ^ { N } S _ { m k } S _ { m k } ^ { * } = \sum _ { m = 1 } ^ { N } \left| S _ { m k } \right| ^ { 2 } = 1
\end{equation}
т.е. сумма квадратов модулей всех матричных элементов любого столбца матрицы рассеяния узла без потерь равна единице.

Второе свойство --- для любой пары столбцов сумма (по строкам) произведений каждого матричного элемента из одного столбца на комплексно сопряженный элемент из той же строки другого столбца равна нулю:
\begin{equation}
	\sum _ { m = 1 } ^ { N } S _ { m l } S _ { m k } ^ { * } = 0 , \quad k \neq l
\end{equation}

\paragraph{Смещение плоскости отсчета.} Предположим, что известна матрица рассеяния при некотором положении плоскости отсчета $z = 0$ в $m$-ом плече узла. При смещении этого сечения на расстояние $l_m$ в направлении распространения падающей волны (по направлению к узлу) новая матрица рассеяния может быть построена по следующим формулам:
\begin{equation}
	\begin{array} { l } { S _ { m k } ^ { \prime } = S _ { m k } e ^ { i \left( h _ { m } l _ { m } + h _ { k } l _ { k } \right) } } \\ { S _ { m m } ^ { \prime } = S _ { m m } e ^ { i 2 h _ { m } l _ { m } } } \end{array}
\end{equation}
где $h_m$ -- постоянная распространения волны в $m$-ом плече. 

Таким образом, изменение плоскости отсчета приводит лишь к изменению фазы коэффициентов матрицы рассеяния, не меняя их абсолютного значения. 


\newpage
\section{Эксперимент}
\subsection{Используемое оборудование}

\begin{enumerate}
	\item СВЧ-генератор Г4-225, в режиме работы на частоте 8.5 ГГц с установленным значением затухания $-4$ дБ
	\item Три шестиполюсника, пластина для закорачивания волновода и две согласованные нагрузки
	\item Измерительная волноводная линия 33-И с кристаллическим детектором, в цепи которого включен амперметр
\end{enumerate}

\subsubsection{Определение коэффициента отражения от нагрузки c помощью измерительной линии}

Измерение элементов матрицы $\Smat$ производится с помощью измерительной линии передачи. Линия передачи позволяет измерить фазу и модуль коэффициента отражения $\Gamma$:

\begin{equation}
	\Gamma=\left(\frac{U_\text{пад}}{U_\text{отр}}\right)_H= | \Gamma | e ^ { i \varphi _ { \mathrm { H } } }
\end{equation}

Модуль коэффициента отражения определялся через коэффициент стоячей волны $K$:
\begin{equation}
	| \Gamma | = \frac { K - 1 } { K + 1 }
\end{equation}

Измерение $K$ производилось путем перемещения вдоль измерительной линии (ИЛ) зонда, показания которого связаны с высокочастотным напряжением в данном сечении линии. 

Для детектирования СВЧ сигнала зонда в ИЛ стоит кристаллический детектор. 
При этом зависимость между током детектора $I$ и приложенным высокочастотным напряжением $|U|$ является нелинейной, и при малых значениях переменного напряжения детектор имеет характеристику $I=f(|U|)$, близкую к квадратичной. В этом случае ток детектора
\begin{equation}
	I=\alpha|U|^2,
\end{equation}
где $\alpha$ --- параметр, зависящий от свойств детектора.

В максимуме и в минимуме распределения поля в линии имеем
\begin{equation}
	I _ { \max } = \alpha | U | _ { \min x } ^ { 2 } , \quad I _ { \min } = \alpha | U | _ { \min } ^ { 2 }
\end{equation}
откуда
\begin{equation}
	\mathrm { K } = \sqrt { \frac { I _ { \mathrm { max } } } { I _ { \mathrm { min } } } }
\end{equation}

Величина фазы коэффициента отражения в сечении $z$ ($z<0$) --- $\psi=\phi_H+2hz$ определялась следующим образом: определялось положение минимума напряжения в ИЛ
% \footnote{При характеристике детектора, близкой к квадратичной, минимум тока детектора никогда не бывает четко выраженным, особенно при небольших К. Для повышения точности отсчета положений минимумов $z_{\min}$ применяют метод <<вилки>>, состоящий в определении двух положений зонда $z_1$ и $z_2$ при одинаковых показаниях индикатора и вычислении $z_{\min}$ по формуле $z_{\min}=(z_1+z_2)/2$.}
 относительно плоскости присоединения нагрузки. Для этого был сначала определен т.н. \textbf{условный конец линии} -- сечение волновода, соответствующее минимуму напряжения при коротком замыкании линии ($z^0_{min}$).
Обозначив расстояние от условного конца линии $z^0_{min}$ до ближайшего минимума напряжения $z_{min}$ \textbf{со стороны генератора} при включенной нагрузке через 
\begin{equation}
	\Delta z _ { \min } \equiv z _ { \min } ^ { 0 } - z _ { \min } = \left| z _ { \min } - z _ { \min } ^ { 0 } \right| =- \left( z _ { \min } - z _ { \min } ^ { 0 } \right)
\end{equation}

И тогда фаза коэффициента отражения (с учетом выражения $h=2\pi/\lambda_B$) определяется формулой
\begin{equation}
	\varphi _ { \mathrm { H } } = 4 \pi \frac { \Delta z _ { \mathrm { min } } } { \lambda _ { \mathrm { B } } } - \pi
\end{equation}

\subsection{Фиксация условного конца линии. Длина волны в волноводе}

Закоротив с помощью короткозамыкателя измерительную линию, зонд измерительной линии установили в ближайший к концу линии узел стоячей волны. 

При этом координата этого узла взята за условный конец линии: 
\begin{equation}
 	z^0_{min}=5.145\text{ см}
 \end{equation}
Измерить длину волны в волноводе $\lambda_B$.

Продвигая зонд в сторону генератора до ближайшего узла, нашли значение длины волны в волноводе:
\begin{equation}
	\frac{\lambda_B}{2}=5.145-2.41=2.735\text{ см}
	\quad\Rightarrow\quad \lambda_B=5.47\text{ см}
\end{equation}

При этом расчетная длина волны $5.49$ см, что неплохо согласуется с экспериментом.

\subsection{Проверка согласованных нагрузок}

Присоединив к скрутке волновода на конце измерительной линии каждую из двух используемых при выполнении работы согласованных нагрузок, определили для них коэффициенты отражения $\Gamma_{H1,2}$. Для точного расчета по приведенным в теории формулам нужно, чтобы они были равны нулю, соответственно для приближенного расчета достаточно близости к нулю. 

Измеренные значения $\Gamma_{H1}=0.0825,\Gamma_{H2}=0.095$, достаточно малы для применения расчетных формул.

\subsection{Измерение параметров шестиполюсников и расчет $S_{km}$}

Для каждого из трех шестиполюсников было осуществлено 6 экспериментов, для измерения диагональных и недиагональных элементов.

\subsubsection{Процедура измерения диагональных элементов}
Пронумеровав условно плечи шестиполюсника как 1,2,3 и зафиксировав их расположение, 
к плечу 1 подключили генератор (Г), а плечи 2 и 3 нагрузили согласованными нагрузками (Н). 

При этом матричное уравнение упрощается, и можно показать, что

\begin{equation}
	\Gamma _ { 11 } = S _ { 11 } = \left| \Gamma _ { 11 } \right| e ^ { i \varphi _ { 11 } }
\end{equation}

Присоединяя генератор поочередно к плечам 2 и 3, определили элементы $S_{22}$ и $S_{33}$.

\subsubsection{Процедура измерения недиагональных элементов}

Для измерения недиагональных элементов, 2 плечо замыкается накоротко, 3 плечо подключается к согласованной нагрузке, и тогда
\begin{equation}
	\Gamma _ { 12 } = S _ { 11 } - \frac { S _ { 12 } S _ { 21 } } { 1 + S _ { 22 } }
\end{equation}
Тогда для произведения недиагональных элементов матрицы рассеяния имеем
\begin{equation}
	S _ { 12 } S _ { 21 } = \left( 1 + S _ { 22 } \right) \left( S _ { 11 } - \Gamma _ { 12 } \right),\qquad\text{1--генератор 2--замыкание 3--нагрузка}
\end{equation}
Поскольку диагональные элементы $S_{mm}$ известны (из предыдущих экспериментов), а коэффициент отражения $\Gamma_{12}=|\Gamma_{12}|e^{i\phi_{12}}$ можно найти с помощью измерительной линии, то этот эксперимент дает возможность определить произведение $S_{12}S_{21}$.
При такой методике измерения невозможно определить отдельно элементы $S_{12}$ и $S_{21}$, однако, если шестиполюсник не содержит невзаимных элементов, то $S_{12} = S_{21}$, и тогда

Расчетные формулы:
\begin{equation}
	S _ { 12 }^2 = \left( 1 + S _ { 22 } \right) \left( S _ { 11 } - \Gamma _ { 12 } \right),\qquad\text{1--генератор 2--замыкание 3--нагрузка}
\end{equation}
\begin{equation}
	S _ { 13 } ^ { 2 } = \left( 1 + S _ { 33 } \right) \left( S _ { 11 } - \Gamma _ { 13 } \right),\qquad\text{1--генератор 2--нагрузка 3--замыкание}
\end{equation}
\begin{equation}
	S _ { 23 } ^ { 2 } = \left( 1 + S _ { 33 } \right) \left( S _ { 22 } - \Gamma _ { 23 } \right),\qquad\text{1--нагрузка 2--генератор 3--замыкание}
\end{equation}

Для расчета корней из комплексных чисел для оптимизации временных затрат использовалась система \textbf{Wolfram Mathematica}.

\newpage
\subsubsection{Результаты измерений и расчётов}

\begin{table}[h!]
	\caption{Измерения характеристик шестиполюсника №1}
	\label{tab:6s1}
	\vspace{1em}
	\centering
	\pgfplotstabletypeset[]{data/tab1.tsv}
\end{table}






\begin{table}[h!]
	\caption{Измерения характеристик шестиполюсника №2}
	\label{tab:6s2}
	\vspace{1em}
	\centering
	\pgfplotstabletypeset[]{data/tab2.tsv}
\end{table}



\begin{table}[h!]
	\caption{Измерения характеристик шестиполюсника №3}
	\label{tab:6s3}
	\vspace{1em}
	\centering
	\pgfplotstabletypeset[]{data/tab3.tsv}
\end{table}


\subsubsection{Полученные матрицы рассеяния}
% \newpage
\begin{equation}
	\pgfplotstableread{data/tab1.tsv}\loadedtable
	\Smat_1 =\mqty(
	\pgfplotstablegetelem{0}{S}\of{\loadedtable}\text{\pgfplotsretval} 
		& \pgfplotstablegetelem{1}{S}\of{\loadedtable}\text{\pgfplotsretval} 
			& \pgfplotstablegetelem{2}{S}\of{\loadedtable}\text{\pgfplotsretval} \\
	\pgfplotstablegetelem{1}{S}\of{\loadedtable}\text{\pgfplotsretval} 
		& \pgfplotstablegetelem{3}{S}\of{\loadedtable}\text{\pgfplotsretval} 
			& \pgfplotstablegetelem{4}{S}\of{\loadedtable}\text{\pgfplotsretval} \\
	\pgfplotstablegetelem{2}{S}\of{\loadedtable}\text{\pgfplotsretval} 
		& \pgfplotstablegetelem{4}{S}\of{\loadedtable}\text{\pgfplotsretval} 
			& \pgfplotstablegetelem{5}{S}\of{\loadedtable}\text{\pgfplotsretval} \\
	)
\end{equation}
\begin{equation}
	\pgfplotstableread{data/tab2.tsv}\loadedtable
	\Smat_2 =\mqty(
	\pgfplotstablegetelem{0}{S}\of{\loadedtable}\text{\pgfplotsretval} 
		& \pgfplotstablegetelem{1}{S}\of{\loadedtable}\text{\pgfplotsretval} 
			& \pgfplotstablegetelem{2}{S}\of{\loadedtable}\text{\pgfplotsretval} \\
	\pgfplotstablegetelem{1}{S}\of{\loadedtable}\text{\pgfplotsretval} 
		& \pgfplotstablegetelem{3}{S}\of{\loadedtable}\text{\pgfplotsretval} 
			& \pgfplotstablegetelem{4}{S}\of{\loadedtable}\text{\pgfplotsretval} \\
	\pgfplotstablegetelem{2}{S}\of{\loadedtable}\text{\pgfplotsretval} 
		& \pgfplotstablegetelem{4}{S}\of{\loadedtable}\text{\pgfplotsretval} 
			& \pgfplotstablegetelem{5}{S}\of{\loadedtable}\text{\pgfplotsretval} \\
	)
\end{equation}
\begin{equation}
	\pgfplotstableread{data/tab3.tsv}\loadedtable
	\Smat_3 =\mqty(
	\pgfplotstablegetelem{0}{S}\of{\loadedtable}\text{\pgfplotsretval} 
		& \pgfplotstablegetelem{1}{S}\of{\loadedtable}\text{\pgfplotsretval} 
			& \pgfplotstablegetelem{2}{S}\of{\loadedtable}\text{\pgfplotsretval} \\
	\pgfplotstablegetelem{1}{S}\of{\loadedtable}\text{\pgfplotsretval} 
		& \pgfplotstablegetelem{3}{S}\of{\loadedtable}\text{\pgfplotsretval} 
			& \pgfplotstablegetelem{4}{S}\of{\loadedtable}\text{\pgfplotsretval} \\
	\pgfplotstablegetelem{2}{S}\of{\loadedtable}\text{\pgfplotsretval} 
		& \pgfplotstablegetelem{4}{S}\of{\loadedtable}\text{\pgfplotsretval} 
			& \pgfplotstablegetelem{5}{S}\of{\loadedtable}\text{\pgfplotsretval} \\
	)
\end{equation}

Полученные матрицы позволяют сказать о наличии потерь, причем первые две матрицы с некоторым приближением можно считать унитарными (сумма квадратов в строках $\sum_m|S_{mk}|^2=0.75\divisionsymbol0.9$), а третья матрица уже не вписывается в такие приближения $\sum_m|S_{mk}|^2=0.02\divisionsymbol0.68$)


\textcolor{white}{
	Lorem ipsum dolor sit amet, consectetur adipisicing elit, sed do eiusmod
	tempor incididunt ut labore et dolore magna aliqua. Ut enim ad minim veniam,
	quis nostrud exercitation ullamco laboris nisi ut aliquip ex ea commodo
	consequat. Duis aute irure dolor in reprehenderit in voluptate velit esse
	cillum dolore eu fugiat nulla pariatur. Excepteur sint occaecat cupidatat non
	proident, sunt in culpa qui officia deserunt mollit anim id est laborum.
	Lorem ipsum dolor sit amet, consectetur adipisicing elit, sed do eiusmod
	tempor incididunt ut labore et dolore magna aliqua. Ut enim ad minim veniam,
	quis nostrud exercitation ullamco laboris nisi ut aliquip ex ea commodo
	consequat. Duis aute irure dolor in reprehenderit in voluptate velit esse
	cillum dolore eu fugiat nulla pariatur. Excepteur sint occaecat cupidatat non
	proident, sunt in culpa qui officia deserunt mollit anim id est laborum.	Lorem ipsum dolor sit amet, consectetur adipisicing elit, sed do eiusmod
	tempor incididunt ut labore et dolore magna aliqua. Ut enim ad minim veniam,
	quis nostrud exercitation ullamco laboris nisi ut aliquip ex ea commodo
	consequat. Duis aute irure dolor in reprehenderit in voluptate velit esse
	cillum dolore eu fugiat nulla pariatur. Excepteur sint occaecat cupidatat non
	proident, sunt in culpa qui officia deserunt mollit anim id est laborum.	Lorem ipsum dolor sit amet, consectetur adipisicing elit, sed do eiusmod
	tempor incididunt ut labore et dolore magna aliqua. Ut enim ad minim veniam,
	quis nostrud exercitation ullamco laboris nisi ut aliquip ex ea commodo
	consequat. Duis aute irure dolor in reprehenderit in voluptate velit esse
	cillum dolore eu fugiat nulla pariatur. Excepteur sint occaecat cupidatat non
	proident, sunt in culpa qui officia deserunt mollit anim id est laborum.	Lorem ipsum dolor sit amet, consectetur adipisicing elit, sed do eiusmod
	tempor incididunt ut labore et dolore magna aliqua. Ut enim ad minim veniam,
	quis nostrud exercitation ullamco laboris nisi ut aliquip ex ea commodo
	consequat. Duis aute irure dolor in reprehenderit in voluptate velit esse
	cillum dolore eu fugiat nulla pariatur. Excepteur sint occaecat cupidatat non
	proident, sunt in culpa qui officia deserunt mollit anim id est laborum.	Lorem ipsum dolor sit amet, consectetur adipisicing elit, sed do eiusmod
	tempor incididunt ut labore et dolore magna aliqua. Ut enim ad minim veniam,
	quis nostrud exercitation ullamco laboris nisi ut aliquip ex ea commodo
	consequat. Duis aute irure dolor in reprehenderit in voluptate velit esse
	cillum dolore eu fugiat nulla pariatur. Excepteur sint occaecat cupidatat non
	proident, sunt in culpa qui officia deserunt mollit anim id est laborum.	Lorem ipsum dolor sit amet, consectetur adipisicing elit, sed do eiusmod
	tempor incididunt ut labore et dolore magna aliqua. Ut enim ad minim veniam,
	quis nostrud exercitation ullamco laboris nisi ut aliquip ex ea commodo
	consequat. Duis aute irure dolor in reprehenderit in voluptate velit esse
	cillum dolore eu fugiat nulla pariatur. Excepteur sint occaecat cupidatat non
	proident, sunt in culpa qui officia deserunt mollit anim id est laborum.	Lorem ipsum dolor sit amet, consectetur adipisicing elit, sed do eiusmod
	tempor incididunt ut labore et dolore magna aliqua. Ut enim ad minim veniam,
	quis nostrud exercitation ullamco laboris nisi ut aliquip ex ea commodo
	consequat. Duis aute irure dolor in reprehenderit in voluptate velit esse
	cillum dolore eu fugiat nulla pariatur. Excepteur sint occaecat cupidatat non
	proident, sunt in culpa qui officia deserunt mollit anim id est laborum.	Lorem ipsum dolor sit amet, consectetur adipisicing elit, sed do eiusmod
	tempor incididunt ut labore et dolore magna aliqua. Ut enim ad minim veniam,
	quis nostrud exercitation ullamco laboris nisi ut aliquip ex ea commodo
	consequat. Duis aute irure dolor in reprehenderit in voluptate velit esse
	cillum dolore eu fugiat nulla pariatur. Excepteur sint occaecat cupidatat non
	proident, sunt in culpa qui officia deserunt mollit anim id est laborum.	Lorem ipsum dolor sit amet, consectetur adipisicing elit, sed do eiusmod
	tempor incididunt ut labore et dolore magna aliqua. Ut enim ad minim veniam,
	quis nostrud exercitation ullamco laboris nisi ut aliquip ex ea commodo
	consequat. Duis aute irure dolor in reprehenderit in voluptate velit esse
	cillum dolore eu fugiat nulla pariatur. Excepteur sint occaecat cupidatat non
	proident, sunt in culpa qui officia deserunt mollit anim id est laborum.	Lorem ipsum dolor sit amet, consectetur adipisicing elit, sed do eiusmod
	tempor incididunt ut labore et dolore magna aliqua. Ut enim ad minim veniam,
	quis nostrud exercitation ullamco laboris nisi ut aliquip ex ea commodo
	consequat. Duis aute irure dolor in re	
}\newpage

% \section{Определение конструкции шестиполюсников}

% Установить возможные конструктивные варианты волноводных узлов, образующих исследуемые шестиполюсники.

% \section{Вывод}
% Изучили основные процессы, происходящие при прохождении сигналов через радиотехнических цепи с нелинейными элементами, эксперементально исследовали характеристики полупроводникового преобразователя частоты и амплитудного диодного детектора. 

\section{Результаты}

Проведя ряд экспериментов, мы определили длину волны в волноводе
\begin{equation}
	\lambda_B=5.47\text{ см},
\end{equation}
при теоретической длине волны $5.49$ см. Для всех шестиполюсников провели по шесть экспериментов, определяя коэффициент отражения в измерительной линии, затем на основании полученных данных рассчитали матрицы рассеяния шестиполюсников: 
\begin{equation}
	\pgfplotstableread{data/tab1.tsv}\loadedtable
	\Smat_1 =\mqty(
	\pgfplotstablegetelem{0}{S}\of{\loadedtable}\text{\pgfplotsretval} 
		& \pgfplotstablegetelem{1}{S}\of{\loadedtable}\text{\pgfplotsretval} 
			& \pgfplotstablegetelem{2}{S}\of{\loadedtable}\text{\pgfplotsretval} \\
	\pgfplotstablegetelem{1}{S}\of{\loadedtable}\text{\pgfplotsretval} 
		& \pgfplotstablegetelem{3}{S}\of{\loadedtable}\text{\pgfplotsretval} 
			& \pgfplotstablegetelem{4}{S}\of{\loadedtable}\text{\pgfplotsretval} \\
	\pgfplotstablegetelem{2}{S}\of{\loadedtable}\text{\pgfplotsretval} 
		& \pgfplotstablegetelem{4}{S}\of{\loadedtable}\text{\pgfplotsretval} 
			& \pgfplotstablegetelem{5}{S}\of{\loadedtable}\text{\pgfplotsretval} \\
	)
\end{equation}
\begin{equation}
	\pgfplotstableread{data/tab2.tsv}\loadedtable
	\Smat_2 =\mqty(
	\pgfplotstablegetelem{0}{S}\of{\loadedtable}\text{\pgfplotsretval} 
		& \pgfplotstablegetelem{1}{S}\of{\loadedtable}\text{\pgfplotsretval} 
			& \pgfplotstablegetelem{2}{S}\of{\loadedtable}\text{\pgfplotsretval} \\
	\pgfplotstablegetelem{1}{S}\of{\loadedtable}\text{\pgfplotsretval} 
		& \pgfplotstablegetelem{3}{S}\of{\loadedtable}\text{\pgfplotsretval} 
			& \pgfplotstablegetelem{4}{S}\of{\loadedtable}\text{\pgfplotsretval} \\
	\pgfplotstablegetelem{2}{S}\of{\loadedtable}\text{\pgfplotsretval} 
		& \pgfplotstablegetelem{4}{S}\of{\loadedtable}\text{\pgfplotsretval} 
			& \pgfplotstablegetelem{5}{S}\of{\loadedtable}\text{\pgfplotsretval} \\
	)
\end{equation}
\begin{equation}
	\pgfplotstableread{data/tab3.tsv}\loadedtable
	\Smat_3 =\mqty(
	\pgfplotstablegetelem{0}{S}\of{\loadedtable}\text{\pgfplotsretval} 
		& \pgfplotstablegetelem{1}{S}\of{\loadedtable}\text{\pgfplotsretval} 
			& \pgfplotstablegetelem{2}{S}\of{\loadedtable}\text{\pgfplotsretval} \\
	\pgfplotstablegetelem{1}{S}\of{\loadedtable}\text{\pgfplotsretval} 
		& \pgfplotstablegetelem{3}{S}\of{\loadedtable}\text{\pgfplotsretval} 
			& \pgfplotstablegetelem{4}{S}\of{\loadedtable}\text{\pgfplotsretval} \\
	\pgfplotstablegetelem{2}{S}\of{\loadedtable}\text{\pgfplotsretval} 
		& \pgfplotstablegetelem{4}{S}\of{\loadedtable}\text{\pgfplotsretval} 
			& \pgfplotstablegetelem{5}{S}\of{\loadedtable}\text{\pgfplotsretval} \\
	)
\end{equation}
\begin{thebibliography}{}
	\bibitem{met} А.С. Зайцева, А.В. Кудрин, Л.Л. Попова. Практикум: Исследование матриц рассеяния волновых узлов. --- Н. Новгород: ННГУ, 2014. --- 22 с.
\end{thebibliography}
\end{document}
